\setcounter{chapter}{98}
\chapter{Esercizi}
\section{Capitolo 5}
\subsection{1}
% Requires: \usepackage{amsmath}
\begin{table}[h]
    \centering
    \begin{tabular}{c c c}
        \underline{Processo} & Durata & priorità \\
        $P_1$ & 10 & 3 \\
        $P_2$ & 1 & 1 \\
        $P_3$ & 2 & 3 \\
        $P_4$ & 1 & 4 \\
        $P_5$ & 5 & 2 \\
    \end{tabular}
    \caption{Processi con durata e priorità}
    \label{tab:processi}
\end{table}
\textbf{FCFS}:
\[
\begin{array}{|c|c|c|c|c|}
  \hline
  \mathbf{P_1} & \mathbf{P_2} & \mathbf{P_3} & \mathbf{P_4} & \mathbf{P_5} \\
  \hline
\end{array}
\]
\textbf{SJF}:
\[\begin{array}{|c|c|c|c|c|}
  \hline
  \mathbf{P_2} & \mathbf{P_4} & \mathbf{P_3} & \mathbf{P_5} & \mathbf{P_1} \\
  \hline
\end{array}\]

\section{Esercizi pre-esame}
ESERCIZIO 2 (5 punti)
In un SO la tabella delle pagine può contenere al massimo 256 (decimale) entry, e
l’offset massimo all’interno di una pagina è FFF (esadecimale).
a) Il SO potrebbe dover adottare un sistema di paginazione a due livelli (motivate la
vostra risposta)?

c) 