\chapter{Memoria di massa}
\section{Disco Rigido}
\subsection{Struttura}
\dfn{Hard Disk (HD)}{Un HD è composto da una serie di piatti o “dischi” sovrapposti, con un diametro che varia tra i 4,5 e i 9 cm.}

\begin{itemize}
    \item Ogni piatto è suddiviso in una serie di tracce circolari concentriche.
    \item Ogni traccia è suddivisa in una serie di settori.
    \item L’insieme delle tracce posizionate nello stesso punto sui vari piatti prende il nome di \textit{cilindro}.
    \item Un “braccio del disco” sostiene una testina di lettura/scrittura per ogni piatto: le testine si muovono tutte simultaneamente e si posizionano sui vari settori del piatto corrispondente (simile al braccio di un giradischi).
\end{itemize}

\nt{
    I settori del disco rappresentano l’unità minima di memorizzazione delle informazioni. Storicamente, ogni settore aveva una dimensione standard di 512 byte; tuttavia, dal 2010 molti produttori hanno aumentato la dimensione fino a 4 KB per settore. Ogni settore memorizza un blocco di dati.
}

\begin{itemize}
    \item I piatti dell'HD ruotano sincronicamente attorno al loro asse, raggiungendo velocità tra 5400 e 15000 RPM (\textit{rounds per minute}), corrispondenti a circa 250 giri al secondo.
    \item Ogni piatto ha associata una testina di lettura/scrittura dei settori, che opera a pochi micron dalla superficie del piatto.
\end{itemize}

\qs{Perché il tempo di accesso a un settore varia?}{
    Una testina può leggere o scrivere su un settore solo quando questo si trova esattamente sotto la testina. Pertanto, il tempo di accesso a un settore dipende principalmente da due componenti:
    \begin{itemize}
        \item \textit{Seek time} (tempo di posizionamento): il tempo necessario affinché la testina raggiunga la traccia contenente il settore desiderato.
        \item \textit{Rotational latency} (latenza rotazionale): il tempo che occorre affinché la rotazione del piatto allinei il settore esatto sotto la testina.
    \end{itemize}
}

\nt{
    A causa della presenza di elementi meccanici, i tempi di accesso sono dell’ordine di alcuni millisecondi.
}

\subsection{Mappatura degli indirizzi}
\dfn{Modello Logico di HD}{
    Un HD può essere logicamente visto come un array unidimensionale di blocchi logici, ciascuno di 512 (o più recentemente 4096) byte: questa è la più piccola unità di trasferimento dati.
}

\begin{itemize}
    \item Ogni settore corrisponde a un blocco logico.
    \item L’array unidimensionale di blocchi logici viene mappato sequenzialmente nei settori del disco:
    \begin{itemize}
        \item Il \textit{settore 0} è il primo settore della traccia più esterna del primo piatto (solitamente in posizione superiore o inferiore nella pila dei piatti).
        \item Successivamente, i settori vengono numerati consecutivamente lungo la traccia fino a raggiungere i settori delle tracce più interne. La numerazione prosegue in modo analogo nei restanti piatti.
    \end{itemize}
\end{itemize}

La mappatura tra blocco logico e settore del disco risulta più complessa di quanto sembri, a causa di due fattori principali:

\begin{itemize}
    \item \textbf{Difetti di fabbricazione:} I dischi possono avere settori difettosi. Tali settori vengono nascosti attraverso il meccanismo di mappatura, che associa blocchi logici a settori funzionanti del disco.
    \item \textbf{Differenze di lunghezza delle tracce:} Non tutte le tracce hanno la stessa lunghezza. 
    \begin{itemize}
        \item Le tracce più lontane dal centro del disco sono più lunghe rispetto a quelle interne e possono contenere fino al 40\% di settori in più.
    \end{itemize}
\end{itemize}

\subsection{Scheduling dei dischi rigidi}

Il sistema operativo (SO) riceve frequentemente richieste di accesso al disco da parte dei processi e deve ottimizzare il trasferimento dei dati per migliorare le prestazioni complessive di accesso al disco.

\nt{
    Il SO non può influenzare la \textit{latenza rotazionale} del disco, che in media corrisponde a metà del tempo necessario per completare una rotazione. Tuttavia, può ridurre il \textit{seek time medio} complessivo ordinando in maniera strategica le richieste in coda, minimizzando così il movimento delle testine.
}

\subsubsection{Algoritmi di scheduling delle richieste I/O}

Esistono diversi algoritmi per gestire lo scheduling delle richieste di I/O del disco. Consideriamo come esempio la seguente sequenza di richieste di accesso, comprese tra la traccia 0 e la traccia 199:

\[
\{ 98, 183, 37, 122, 14, 124, 65, 67 \}
\]

\nt{
    Le tracce potrebbero trovarsi su piatti diversi, dato che tutti i piatti ruotano insieme e le testine si muovono simultaneamente. Tuttavia, per semplicità possiamo supporre l’esistenza di un unico piatto e che la testina sia inizialmente posizionata sulla traccia (o cilindro) numero 53.
}

\begin{figure}[h] \centering \includegraphics[width=0.25\linewidth]{images/scheduling_hdd_98.png}
\caption{Coda delle richieste: 98, 183, 37, 122, 14, 124, 65, 67 In tutto la testina attraversa 640 tracce. Invece che “122 - 14124” era meglio fare “122 - 124 - 14” }
\end{figure}

\subsubsection{C-SCAN (Circular-SCAN)}
Fornisce un tempo di attesa per le varie richieste più uniforme di altri algoritmi, anche se non riesce a garantire un tempo medio di attesa minimo.\\
La testina si muove da un estremo all’altro del piatto, servendo le richieste.\\
Quando raggiunge l’estremità del piatto, torna immediatamente all’inizio senza servire richieste.\\
In pratica, \textbf{tratta i settori/cilindri come una lista circolare.}\\

\begin{figure}[h] \centering \includegraphics[width=0.25\linewidth]{images/scheduling_hdd_cscan.png}
    \caption{Coda delle richieste: 98, 183, 37, 122, 14, 124, 65, 67 La testina attraversa 183 tracce + 200 tracce per tornare indietro, ma questo ritorno richiede poco tempo, perché la
    testina non deve mai fermarsi e ripartire}
\end{figure}

\nt{Questo è l'algoritmo di scheduling più utilizzato nei sistemi operativi moderni.}


\section{Formattazione del disco}
Prima di poter essere utilizzato, un disco rigido deve essere sottoposto a un processo di \textit{formattazione}, che avviene in due fasi principali:

\begin{itemize}
    \item Questa operazione viene solitamente effettuata dal costruttore dell’HD e ha lo scopo di:
    \begin{itemize}
        \item Associare un numero univoco a ogni settore.
        \item Allocare uno spazio per un codice di correzione degli errori (ECC), utilizzato durante ogni operazione di I/O su quel settore.
    \end{itemize}
    \item Durante questa fase è possibile definire la dimensione dei blocchi fisici, ad esempio 512 byte o 4096 byte per settore.
\end{itemize}

\subsubsection{Formattazione logica}

\begin{itemize}
    \item Questo processo, gestito dal sistema operativo, è necessario per creare e organizzare il File System.
    \item Il sistema operativo esegue le seguenti operazioni:
    \begin{itemize}
        \item Creazione della lista dei blocchi liberi secondo lo schema adottato.
        \item Creazione di una directory iniziale, punto di partenza per l’intera struttura del File System.
        \item Riservazione di aree specifiche per la gestione diretta da parte del SO:
        \begin{itemize}
            \item \textbf{Boot Block:} il blocco di avviamento.
            \item \textbf{Aree per gli attributi dei file:} ad esempio, gli \textit{index-node} in Unix o la MFT (\textit{Master File Table}) in Windows.
        \end{itemize}
    \end{itemize}
\end{itemize}

\subsection{Il Boot Block}

\begin{itemize}
    \item Contiene il codice necessario per avviare il sistema operativo.
    \item All’accensione, un piccolo programma residente in ROM istruisce il \textit{disk controller} a trasferire il contenuto del Boot Block nella RAM.
    \item Una volta trasferito, il controllo passa al codice del Boot Block, che avvia l’intero sistema operativo caricandolo dal disco stesso.
\end{itemize}

\section{Gestione dell'area di SWAP}
Durante la formattazione logica del disco rigido, il sistema operativo riserva uno spazio per l'\textit{area di Swap}, che funge da memoria virtuale utilizzata per lo scambio di pagine o segmenti tra RAM e memoria secondaria.

\subsection{Gestione dell'area di Swap}

\begin{itemize}
    \item \textbf{Swap come file:} 
    \begin{itemize}
        \item Nel caso più semplice, l’area di Swap può essere un file di grandi dimensioni all’interno del File System.
        \item Nei sistemi Windows, lo \textit{swap file} è denominato \texttt{pagefile.sys}. 
        \item Gli utenti possono regolarne la dimensione, ad esempio riducendola in presenza di una grande quantità di RAM, per recuperare spazio sul disco rigido.
    \end{itemize}

    \item \textbf{Swap come partizione dedicata:}
    \begin{itemize}
        \item Una porzione specifica del disco rigido, chiamata \textit{partizione di Swap}, può essere riservata esclusivamente a questo scopo.
        \item Questa partizione è gestita diversamente rispetto a un normale File System, adottando strategie di allocazione ottimizzate per la velocità di accesso.
        \item Ad esempio, i blocchi possono essere allocati in modo contiguo per evitare la ricerca di blocchi liberi, riducendo così il tempo necessario per lo scambio.
    \end{itemize}
\end{itemize}

\subsubsection{Dimensionamento dell'area di Swap}

\nt{
    È fondamentale dimensionare adeguatamente l’area di Swap per garantire che il sistema operativo trovi rapidamente uno spazio libero per lo scambio di pagine e segmenti.
}

\begin{itemize}
    \item \textbf{Consigli per il dimensionamento:}
    \begin{itemize}
        \item In Solaris, si raccomanda di dimensionare l’area di Swap in base alla differenza tra lo spazio di indirizzamento logico e quello fisico.
        \item In Linux, si suggerisce di utilizzare un’area di Swap pari al doppio della RAM disponibile.
    \end{itemize}
    
    \item \textbf{Sistemi con più dischi:}
    \begin{itemize}
        \item In configurazioni multi-disco, è possibile creare un’area di Swap per ciascun disco.
        \item Ciò consente di sfruttare le aree di Swap in parallelo, bilanciando il carico di lavoro e migliorando le prestazioni.
    \end{itemize}
\end{itemize}


\section{Sistemi RAID}
Gli hard disk (HD) e i dischi a stato solido (SSD) sono dispositivi notevolmente più lenti rispetto al processore e alla memoria primaria. Inoltre, il guasto di un disco rigido rappresenta un rischio significativo: in assenza di un back-up, i dati memorizzati possono essere irrimediabilmente persi o, nel migliore dei casi, non disponibili durante i tempi di riparazione.

\subsection{Introduzione ai sistemi RAID}

\dfn{RAID (Redundant Array of Independent Disks)}{
    È un sistema di configurazione della memoria secondaria progettato per migliorare sia le prestazioni sia l'affidabilità degli hard disk. 
}

\begin{itemize}
    \item RAID si rivela utile in ogni settore, ma è essenziale in contesti critici dove il servizio non può mai interrompersi, come nel settore finanziario e bancario.
    \item Il concetto di RAID fu introdotto nel 1988 da Patterson, Gibson e Katz, con l'acronimo iniziale \textit{Redundant Array of Inexpensive Disks}, successivamente ridefinito come \textit{Redundant Array of Independent Disks}.
    \item La controparte dei sistemi RAID è rappresentata da dispositivi SLED (\textit{Single Large Expensive Disk}).
\end{itemize}

\subsection{Caratteristiche principali di un sistema RAID}

\begin{itemize}
    \item Un sistema RAID è costituito da un insieme di dischi (detto \textit{disk array}) che viene visto dal sistema operativo come un singolo dispositivo di memorizzazione, più veloce e affidabile di un SLED.
    \item La gestione dei dischi è demandata al \textit{controller del RAID}, che si occupa di distribuire i dati secondo criteri specifici, senza necessità di modifiche al sistema operativo. 
    \item Questo vantaggio semplifica notevolmente la vita degli amministratori di sistema (\textit{system administrators}).
\end{itemize}

\subsection{Idee principali alla base di RAID}

Le due idee fondamentali di un sistema RAID sono:
\begin{enumerate}
    \item \textbf{Distribuzione dei dati:} L'informazione è suddivisa su più dischi per parallelizzare parte delle operazioni di accesso e migliorare le prestazioni.
    \item \textbf{Ridondanza dei dati:} L'informazione è duplicata su più dischi. In caso di guasto di un disco, il sistema può continuare a funzionare, recuperando i dati dal disco di backup.
\end{enumerate}

\subsection{Livelli di RAID}

Differenti schemi di implementazione delle idee sopra descritte hanno portato alla definizione di vari livelli RAID, numerati da 0 a 6. 
\nt{
    Da qui in poi, ci si discosta parzialmente dalla trattazione del libro di testo.
}
\subsection{RAID di Livello 0}

\nt{
    I sistemi RAID di livello 0, pur essendo inclusi nella famiglia RAID, non offrono ridondanza dei dati, quindi non aumentano l'affidabilità del sistema.
}

\dfn{RAID Livello 0}{
    Un'architettura RAID in cui il disco virtuale (cioè l'insieme di blocchi logici consecutivi visti dal sistema operativo) viene suddiviso in \textit{strip} (strisce) di $k$ blocchi consecutivi ciascuna. Questa tecnica, chiamata \textbf{striping}, distribuisce i dati su più dischi per migliorare le prestazioni.
}

\begin{itemize}
    \item Ogni \textit{strip} è identificata da un numero e contiene $k$ blocchi consecutivi.
    \begin{itemize}
        \item Lo strip 0 contiene i blocchi da $0$ a $k-1$.
        \item Lo strip 1 contiene i blocchi da $k$ a $2k-1$.
    \end{itemize}
    \item Gli strip sono distribuiti sui dischi disponibili secondo la formula:
    \[
    \text{numero-strip} \mod \text{dischi-nel-sistema}
    \]
    \item Ad esempio, in un sistema con 4 dischi:
    \begin{itemize}
        \item Il disco 0 conterrà gli strip 0, 4, 8, ...
        \item Il disco 1 conterrà gli strip 1, 5, 9, ...
    \end{itemize}
    \item Se $k=1$, ogni strip contiene un singolo blocco. In questo caso:
    \begin{itemize}
        \item Il blocco 0 sarà sul primo settore del primo disco.
        \item Il blocco 1 sarà sul primo settore del secondo disco.
        \item Il blocco 4 sarà sul secondo settore del primo disco, e così via.
    \end{itemize}
\end{itemize}

\ex{Esempio di richiesta su un RAID Livello 0}{
    Supponiamo che il sistema operativo richieda la lettura di dati contenuti negli strip 4, 5, 6 e 7. 
    \begin{itemize}
        \item Il controller RAID suddividerà la richiesta in quattro letture parallele, una per ciascun disco.
        \item Su un singolo disco SLED, invece, tutti i settori dei quattro strip dovrebbero essere letti in sequenza.
    \end{itemize}
    Di conseguenza, l'operazione sarà completata più rapidamente con il RAID 0.
}

\clm{Prestazioni e limiti del RAID Livello 0}{}{
    \begin{itemize}
        \item \textbf{Miglioramenti delle prestazioni:} Un RAID di livello 0 è particolarmente efficiente quando le richieste coinvolgono molti strip consecutivi, e il sistema è composto da un elevato numero di dischi.
        \item \textbf{Limitazioni:}
        \begin{itemize}
            \item Richieste che riguardano un singolo strip non ottengono alcun miglioramento rispetto a un disco SLED.
            \item L'affidabilità del sistema è inferiore a quella di un singolo disco SLED, poiché il \textit{Mean Time To Failure} (MTTF) complessivo diminuisce con l'aumento del numero di dischi.
        \end{itemize}
        \item \textbf{Applicazioni tipiche:} Il RAID 0 è utilizzato in applicazioni che richiedono alte prestazioni ma non necessitano di particolare affidabilità, come lo streaming audio e video.
    \end{itemize}
}
\begin{figure}[h] \centering \includegraphics[width=0.50\linewidth]{images/raid_livelloZero.png} \caption{Raid Livello Zero} \label{fig:2-23a} \end{figure}

\subsection{RAID di Livello 1: Mirroring}

\dfn{RAID Livello 1}{
    Un'architettura RAID in cui ogni disco di dati $D$ è affiancato da un disco di mirroring che contiene una copia esatta dei dati memorizzati in $D$. La configurazione più semplice prevede l'utilizzo di due dischi: uno contenente i dati e l'altro la copia esatta.
}

\begin{itemize}
    \item La duplicazione dei dati garantisce una maggiore affidabilità:
    \begin{itemize}
        \item Se un disco si rompe, il sistema utilizza il disco di mirroring senza perdita di dati o interruzione del servizio.
    \end{itemize}
    \item \textbf{Prestazioni:} 
    \begin{itemize}
        \item La scrittura dei dati è leggermente rallentata, poiché deve avvenire contemporaneamente su due dischi.
        \item La lettura è più veloce, dato che i dati possono essere letti da entrambi i dischi in parallelo.
    \end{itemize}
    \item \textbf{Limiti:} Il costo di implementazione è elevato, poiché richiede il raddoppio del numero di dischi.
\end{itemize}

\begin{figure}[h] \centering \includegraphics[width=0.20\linewidth]{images/raid_mirro1.png} \caption{Raid Mirror} \end{figure}

\ex{Combinazione RAID Livello 0+1}{
    Unendo le caratteristiche dello striping (RAID 0) e del mirroring (RAID 1), si ottengono configurazioni RAID ibride che massimizzano sia le prestazioni che l'affidabilità.
}

\subsection{RAID di Livello 01: Striping + Mirroring}

\dfn{RAID Livello 01}{
    Un'architettura RAID che combina lo striping (livello 0) e il mirroring (livello 1). 
    \begin{itemize}
        \item I dati sono suddivisi in \textit{strip} distribuiti su più dischi (come nel RAID 0).
        \item Ogni disco è duplicato da un disco di mirroring (come nel RAID 1).
    \end{itemize}
}

\begin{itemize}
    \item \textbf{Affidabilità:} 
    \begin{itemize}
        \item Quando un disco si rompe, il sistema RAID utilizza il disco di mirroring per accedere ai dati, mantenendo il sistema operativo attivo.
        \item Un disco rotto può essere sostituito, e i dati saranno copiati automaticamente dal "gemello" (se è disponibile uno \textit{spare disk}).
    \end{itemize}
    \item \textbf{Prestazioni:}
    \begin{itemize}
        \item La lettura di un blocco di dati che coinvolge $n$ strip può essere eseguita con $n$ letture in parallelo, sfruttando sia i dischi primari sia quelli di mirroring.
    \end{itemize}
    \item \textbf{Costo:} È una delle soluzioni RAID più costose, poiché richiede la duplicazione di tutti i dischi.
\end{itemize}

\begin{figure}[h] \centering \includegraphics[width=0.50\linewidth]{images/raid_mirror01.png} \caption{Raid Mirro 0+1} \end{figure}

\clm{Applicazioni del RAID 01}{}{
    Il RAID di livello 01 è ideale per contesti in cui l'affidabilità è fondamentale, come la gestione di dati finanziari o bancari. 
    \begin{itemize}
        \item Aumentando il numero di dischi coinvolti, migliorano le prestazioni complessive del sistema.
    \end{itemize}
}

\subsection{RAID Livello 10 (Mirroring + Striping)}

\dfn{RAID Livello 10}{
    Una configurazione RAID che combina il mirroring e lo striping in modo inverso rispetto al livello 01:
    \begin{itemize}
        \item Il mirroring viene eseguito per primo, creando coppie di dischi speculari.
        \item Successivamente, viene applicato lo striping alle coppie di dischi.
    \end{itemize}
}

\begin{figure}[h] \centering \includegraphics[width=0.50\linewidth]{images/raid_mirror10.png} \caption{Raid 10} \end{figure}

\begin{itemize}
    \item \textbf{Prestazioni:} Simili a quelle del RAID livello 01.
    \item \textbf{Affidabilità:} 
    \begin{itemize}
        \item Vantaggi nel recupero dei dati in caso di guasti, poiché ogni coppia di dischi può essere gestita indipendentemente.
    \end{itemize}
\end{itemize}

\subsection{RAID Livello 4 (Striping con Parità)}

\dfn{RAID Livello 4}{
    Una configurazione RAID che utilizza lo striping a livello di blocchi e calcola uno strip di parità per consentire il recupero dei dati in caso di guasto.
    \begin{itemize}
        \item Gli strip di parità sono memorizzati in un disco dedicato.
        \item Ogni strip di parità è calcolato usando i corrispondenti strip degli altri dischi.
    \end{itemize}
}

\begin{itemize}
    \item \textbf{Vantaggi:}
    \begin{itemize}
        \item Risparmio di dischi rispetto al RAID 01.
    \end{itemize}
    \item \textbf{Svantaggi:}
    \begin{itemize}
        \item Scritture più lente, poiché ogni modifica richiede l'aggiornamento dello strip di parità.
        \item Il disco di parità può diventare un collo di bottiglia, aumentando la probabilità di guasti.
    \end{itemize}
\end{itemize}

\begin{figure}[h] \centering \includegraphics[width=0.50\linewidth]{images/raid_level4.png} \caption{Raid livello 4} \end{figure}

\subsection{RAID Livello 5 (Striping con Parità Distribuita)}

\dfn{RAID Livello 5}{
    Una configurazione RAID simile al livello 4, ma con gli strip di parità distribuiti tra tutti i dischi, riducendo il carico sul disco di parità.
    \begin{itemize}
        \item Ogni disco contiene sia strip di dati sia strip di parità.
    \end{itemize}
}

\begin{itemize}
    \item \textbf{Vantaggi:}
    \begin{itemize}
        \item Migliore distribuzione del carico di scrittura.
        \item Ottima combinazione di prestazioni, affidabilità e capacità di memorizzazione.
    \end{itemize}
    \item \textbf{Limiti:}
    \begin{itemize}
        \item Ricostruzione complessa in caso di guasto, data la distribuzione degli strip di parità.
    \end{itemize}
    \item \textbf{Utilizzo:} Livello RAID più usato per applicazioni generiche.
\end{itemize}

\begin{figure}[h] \centering \includegraphics[width=0.50\linewidth]{images/raid_level5.png} \caption{Raid Livello 5} \end{figure}

\subsection{RAID Livello 6 (Striping con Doppia Parità Distribuita)}

\dfn{RAID Livello 6}{
    Una configurazione RAID simile al livello 5, ma con due livelli di parità distribuiti tra i dischi, permettendo di resistere al guasto simultaneo di due dischi.
}

\begin{itemize}
    \item \textbf{Vantaggi:}
    \begin{itemize}
        \item Maggiore affidabilità rispetto al RAID 5.
    \end{itemize}
    \item \textbf{Svantaggi:}
    \begin{itemize}
        \item Richiede un disco in più rispetto al RAID 5 per memorizzare la stessa quantità di dati.
        \item Maggiore overhead computazionale.
    \end{itemize}
    \item \textbf{Utilizzo:} Poco usato, poiché la rottura contemporanea di due dischi è un evento raro.
\end{itemize}

\subsection{Ricostruzione di Dati Persi con la Parità}

\dfn{Parità}{
    Una tecnica che consente di recuperare i dati di un disco guasto usando la parità bit a bit calcolata con un'operazione EX-OR tra le stringhe binarie memorizzate nei vari dischi.
    \begin{itemize}
        \item La parità è definita come:
        \[
        \text{parità} = \text{stringa}_0 \oplus \text{stringa}_1 \oplus \cdots \oplus \text{stringa}_m
        \]
        \item Se un disco si rompe, il suo contenuto può essere ricostruito:
        \[
        \text{stringa}_j = \text{parità} \oplus \text{stringa}_0 \oplus \text{stringa}_1 \oplus \cdots \oplus \text{stringa}_{j-1} \oplus \text{stringa}_{j+1} \oplus \cdots \oplus \text{stringa}_m
        \]
    \end{itemize}
}

\ex{Esempio di Recupero con Parità}{
    Consideriamo tre stringhe binarie di 8 bit:
    \begin{align*}
    a & = 01100011 \\
    b & = 10101010 \\
    c & = 11001010 \\
    \end{align*}
    La parità calcolata è:
    \[
    p = a \oplus b \oplus c = 00000011
    \]
    Se si perde la stringa $a$, possiamo ricostruirla:
    \[
    a = p \oplus b \oplus c = 01100011
    \]
}

\section{Memorie a Stato Solido (SSD)}

\dfn{Memorie a Stato Solido}{
    Memorie permanenti e riscrivibili basate su tecnologia flash, utilizzate come supporto di memoria secondaria. Caratteristiche principali:
    \begin{itemize}
        \item Organizzate in pagine da 2 a 16 Kbyte.
        \item Operazioni di lettura/scrittura coinvolgono l'intera pagina.
        \item Limite di circa 100.000 riscritture per pagina.
        \item Prima della riscrittura, le pagine devono essere cancellate tramite un processo detto \textit{flashing}.
    \end{itemize}
}

\begin{itemize}
    \item \textbf{Utilizzo:}
    \begin{itemize}
        \item Dispositivi mobili: smartphone, tablet.
        \item Computer, spesso nei portatili di fascia alta:
        \begin{itemize}
            \item Come sostituto del disco rigido per aumentare velocità e leggerezza.
            \item Come complemento al disco rigido.
        \end{itemize}
    \end{itemize}
    \item \textbf{Tipi di supporto:}
    \begin{itemize}
        \item \textbf{SSD (Solid State Disk):} Montati su schede o contenitori installabili nel computer.
        \item \textbf{Pen Drive:} Incapsulati in contenitori rimuovibili con connessione USB.
    \end{itemize}
\end{itemize}

\subsection{Confronto tra Tipi di Memorie}

\begin{table}[h!]
    \centering
    \begin{tabular}{|l|c|c|c|}
        \hline
        \textbf{Caratteristica} & \textbf{Hard Disk} & \textbf{Flash Memory} & \textbf{RAM} \\
        \hline
        \textbf{Costo per GB} & $\sim$ 0,015\$ & $\sim$ 0,05-0,1\$ & $\sim$ 2-4\$ \\
        \hline
        \textbf{Velocità in Lettura (T = RAM)} & $\sim$ 1000 × T & $\sim$ 4 × T & T \\
        \hline
        \textbf{Velocità in Scrittura (T = RAM)} & $\sim$ 1000 × T & 10-100 × T & T \\
        \hline
    \end{tabular}
    \caption{Confronto tra Hard Disk, Memoria Flash e RAM.}
    \label{tab:memorie}
\end{table}

\subsection{Utilizzo degli SSD}

\begin{itemize}
    \item \textbf{Sostituzione dell'hard disk:}
    \begin{itemize}
        \item Maggiore velocità di accesso.
        \item Minore peso per dispositivi portatili.
    \end{itemize}
    \item \textbf{Combinazione con hard disk:}
    \begin{itemize}
        \item \textit{Cache permanente:} Collocata tra l'hard disk e la RAM.
        \item \textit{Secondo hard disk:} Utilizzato per:
        \begin{itemize}
            \item Sistema operativo.
            \item Applicativi.
            \item File più utilizzati e area di swap.
        \end{itemize}
        Gli altri file rimangono memorizzati sull'hard disk.
    \end{itemize}
    \item \textbf{Politica di Accesso:}
    \begin{itemize}
        \item Accesso completamente diretto.
        \item Di solito gestito con politica \textit{First Come, First Served (FCFS)}.
    \end{itemize}
\end{itemize}
