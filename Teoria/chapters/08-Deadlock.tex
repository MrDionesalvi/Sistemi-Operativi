\chapter{Stallo dei processi (deadlock)}
\nt{Questo capitolo è \textbf{facoltativo}, presente per dare più integrità agli appunti totali}

\section{Definizione}
\dfn{}{Situazione in cui ciascun processo in un insieme di $n$ processi ($n \geq 2$) si trova in uno stato di \textit{attesa} per il verificarsi di un evento che solo uno degli aaltri processi dell'insieme può provocare}
Il risultato è, chiaramente, una attesa infinita da parte di tutti gli $n$ processi!

\section{Situazioni simili anche nella realtà}:
\textit{When two trains approach each other at a crossing, both shall come to a full stop and neither shall start up again until the other has gone}\\
I SO di oggi non affrontano il problema, ma questo spetta a gli utenti. In futuro chi lo sa potrà diventare un compito dei SO.

\section{Problema dei nastri}
\subsection{Dati Condivisi}
\begin{verbatim}
semaphore avail = 2; // il semaforo controlla la disponibilità dei nastri
\end{verbatim}
\subsection{Codice dei Processi P1 e P2}

\begin{verbatim}
// Processo P1
begin
    // codice preliminare
    wait(avail); // P1 prende il primo nastro
    // altre operazioni
    wait(avail); // P1 tenta di prendere il secondo nastro
    // utilizza i nastri
    signal(avail); // P1 rilascia il primo nastro
    signal(avail); // P1 rilascia il secondo nastro
    // codice finale
end

// Processo P2
begin
    // codice preliminare
    wait(avail); // P2 prende il primo nastro
    // altre operazioni
    wait(avail); // P2 tenta di prendere il secondo nastro
    // utilizza i nastri
    signal(avail); // P2 rilascia il primo nastro
    signal(avail); // P2 rilascia il secondo nastro
    // codice finale
end
\end{verbatim}

\subsection{Problema di Deadlock}
Questo scenario può portare a un deadlock: se entrambi i processi eseguono il primo \texttt{wait(avail)} e occupano ciascuno un nastro, nessuno dei due sarà in grado di eseguire il secondo \texttt{wait(avail)} perché il semaforo \texttt{avail} è inizializzato a 2. Di conseguenza, entrambi i processi rimarranno bloccati.

\section{Un ponte ad una sola corsia}
Ciascuna posizione di marcia può essere vista come una \textbf{risorsa}, una situazione di deadlock può essere risolta se un auto \textbf{torna indietro} $\Longrightarrow$ (libera una risorsa già occupata), si verifica \textbf{starvation} se ciascuna auto sul ponte attende che l'altra liberi l'unica corsia di marcia

\section{Modello del sistema}
Un sistema (HW + SO) può essere visto come formato da:
\begin{itemize}
    \item un insieme finito di tipi di risorse $R$ (cicli di CPU, spazio di memoria, device di I/O),
    \item Ogni tipo di risorsa è formata da un certo numero di istanze indistinguibili fra loro (ad esempio la RAM può essere divisa in porzioni identiche, ciascuna delle quali può ospitare un processo),
    \item Un insieme di processi $P$ che hanno bisogno di una o più istanze di alcune delle risorse per portare a termine la computazione.
\end{itemize}

\subsection*{Definizione di Deadlock}

Si definisce \textbf{deadlock} di un sottoinsieme di processi del sistema $\{P_1, P_2, \dots, P_n\} \subseteq P$ la situazione in cui ciascuno degli $n$ processi $P_i$ è in attesa del rilascio di una risorsa detenuta da uno degli altri processi del sottoinsieme; si forma cioè una catena circolare per cui:

\[
P_1 \text{ aspetta } P_2 \dots \text{ aspetta } P_n \text{ aspetta } P_1
\]
\noindent Anche se non tutti i processi del sistema sono bloccati, la situazione non è desiderabile in quanto può bloccare alcune risorse e danneggiare anche i processi non coinvolti nel deadlock.

\section{Caratterizzazione dei Deadklock}
Il SO può avvalersi di una opportuna rappresentazione detta \textbf{grafo} di assegnazione delle risorse che in ogni istante registra quali risorse sono assegnate a quale processo, e quali risorse sta \textbf{aspettando} ciascun processo.
\section{Metodi per prevenire dei Deadlock}
\begin{enumerate}
    \item Prevenire o evitare i deadlock, usando un opportuno protocollo di richiesta e assegnamento delle risorse
    \item lasciare che il deadlock si verifichi, ma fornire strumenti per la scoperta e il recupero dello stesso, esplorando il grafo di assegnazione delle risorse alla ricerca di cicli.
    \begin{itemize}
        \item \clm{}{}{Tuttavia, la soluzione 1 genera un eccessivo sottoutilizzo delle risorse, mentre la soluzione 2 non evita il problema e richiede lavoro al SO per eliminare il deadlock, dunque i SO moderni adottano la soluzione 3: }
    \end{itemize}
    \item \textbf{Lasciare agli utenti la prevenzione/gestione dei deadlock}
\end{enumerate}